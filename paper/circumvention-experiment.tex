\documentclass{template}
\usepackage{color}
\usepackage[hyphens]{url}
\usepackage{longtable}
\usepackage{graphicx}
\usepackage{enumitem}
\usepackage{pdfpages}
\usepackage{hyperref}

\def\etal{{\it et al.~}}
\newenvironment{packed_enum}{
\begin{enumerate}
  \setlength{\itemsep}{1pt}
  \setlength{\parskip}{0pt}
  \setlength{\parsep}{0pt}
}{\end{enumerate}}
\newenvironment{packed_item}{
\begin{itemize}
  \setlength{\itemsep}{1pt}
  \setlength{\parskip}{0pt}
  \setlength{\parsep}{0pt}
}{\end{itemize}}

\begin{document}

\title{Tor's Usability for Censorship Circumvention}
\numberofauthors{1}
\author{
 \alignauthor Linda N. Lee, David Fifield, Nathan Malkin \\
   \vspace{0.5em}
   \affaddr{University of California, Berkeley} \\
   \affaddr{\{lnl,fifield,nmalkin\}@cs.berkeley.edu}\\
}
\maketitle

\begin{abstract}
{\color{red}
Tor has grown beyond its original purpose as and has 
since become an important Internet circumvention tool.
We specifically examine it usability as a censorship circumvention tool,
an essential facet for adoption and use.  
We focus our analysis on the connection configuration dialog of Tor browser,
as censorship circumvention requires correct transport configurations.
Our talk will describe a future study aimed at evaluating 
if and how easily users can circumvent censorship using Tor Browser,
isolating specific browser features to study in the process. 
To this end, we will conduct a large-scale user study examining hundreds of users 
on how they navigate Tor's configuration wizard to complete seven browsing tasks 
in three different adversarial settings. 
We solicit feedback to improve our study's design.}
\end{abstract}

\keywords{Censorship, Security, User Studies, Anonymity, Tor}

\section{Tor's Usability}
{\color {red}
%Idea is to cover background on Tor's usability as an anonymity tool, and then highlight the shift in focus to evaluating Tor's usability as a censorship circumvention tool. 

Tor is the most widely used anonymity tool today. However, there are complaints that Tor is not usable. Norcie~\cite{norcie2012eliminating} did an experiment which identified stopping points in Tor browser, 
finding out when people would get frustrated with Tor so much they would stop using it. 
%redo this next couple sentences into something better
But what about the people who are using Tor? what usability issues would there be if people were not stopped? 
To our knowledge, that has been the only user study of Tor. 
Since then, Tor has had a lot of updates. 
Additionally, there were a lot of features that were untested. 
There have been been no major usability evaluations of
Tor Browser since the introduction of the 4.0 series,
which introduced radical UI changes.

We briefly describe the results of a completed study
that examined the download and installation processes, 
as well as basic browsing behaviors.
This study uncovered a number of bugs and stopping points
which has already effected concrete change in the browser.

We ran a small pilot study of five journalists which involved a cognitive walk through in which 
participants explained the motivation for their actions and any confusion that they had during the process. 
We did this to find new stopping points, and sources of confusion. 
But in the process, we were also able to observe how users would download Tor, 
if they could understand the address bar/menu, how well they were able to complete basic tasks
(searching, setting new circuits, etc.), and generally if they had any usability complaints about the browser. 

We recruited five journalists from Berkeley by reaching out to journalist contacts. 
%Demographics here. 
During the study, we made a video recording of each participant's computer screen
and simultaneously projected it
into another room with Tor developers.
We recorded what they were saying out loud in their cognitive walk through,
and later added these words to the screen videos as subtitles.
The participants also took an exit survey which estimated their familiarity with technology and security. On average, participants took 26 minutes to complete the study ($\sigma = 7$~min).  The result was that people did have difficulty with installing Tor Browser (principally because of the Gatekeeper code-signing feature on OS~X), did not understand what many of the many options meant, and were confused about why certain things were happening. 
Our talk will feature brief highlights of the screen videos and a summary of interface changes.

After our first  usability evaluation of Tor, it was clear to us that so many of the features had been left unevaluated---such as advanced web browsing tasks, the configuration menu, automatic updates, and identity and cookie management. 
%say something about how we resolved this somehow, and how it doesn't necessarily need our attention now
We found that the most effective way to resolve the problems encountered was more user guidance and interface remodeling rather than continued user observations.
%emphasize how we are aiming for new research, not new test cases for debugging an interface
Rather than selecting the features to study in isolation, 
we decided to focus on an important use case of Tor browser, censorship circumvention. 
%say something along the lines of ``x people are using it for this, and there's no work on it,'' 
%if there are stats on this type of thing.
To our knowledge, this is the first user study investigating the usability of Tor as a 
censorship circumvention tool, rather than an anonymity tool. }



\section{Design}

\noindent {\bfseries Overview}
{\color {red}
% test tor, test improvements, may do followups on demand, push changes.

% setting up the environment wrt interface paths.


%improve what is below:
For users to circumvent censorship in their resident countries, they will need to configure Tor to set up a proxy, bridge, or both. There is a configuration wizard to help guide users through setting up their connection, but our hypothesis is that the average user will not easily be able to configure Tor correctly in these adverse settings, as they require the user to provide IP addresses of proxies or bridges to use. The goal of our experiment is to see how successful users are at carrying out common browsing tasks in an adversarial setting using Tor.}\\

\begin{figure*}[t]
  \centering
    \includegraphics[width=\textwidth]{../torconfig.png}
    \caption{This flow chart shows all of the possible paths taken through the Tor configuration interface. A state represents a window in the interface, with the exception of the leaves, which indicate the final action taken by the interface (``q'' for quit, ``s'' for success, and ``e'' for error). The transitions between the states indicate which user action causes that transition. The cause for error is can be found \href{https://github.com/lindanlee/circumvention-ux-tor/blob/master/torconfig.dot}{here}.}
\end{figure*}

\begin{figure}[t]
\label{fig:bridges}
  \centering
    \includegraphics[width=0.5\textwidth]{configuration-screenshot.png}
    \caption{The Tor Bridges Configuration window of the Tor configuration interface. Tor users in censored environments and some of our participants will be required to select the correct bridge for circumventing censorship. Note that the interface prompts users to choose a transport type, which can be the source of confusion. Bridges are non-listed guard relays to the Tor network, and some of them can have transport types which obfuscate traffic in different ways.}
\end{figure}


\noindent {\bfseries Experiment Logistics} 
The IRB protocol to run this user study has been approved (2014-12-6995). 
We plan to recruit from 100-200 users for the purpose of this study, 
making it largest user study to examine Tor.  This study will be conducted at the 
Experimental Social Science Laboratory (Xlab)
at the University of California, Berkeley, which consists of 36 laptops with 
cubicle walls separating each laptop. We will use individual host firewalls to simulate
censorship environments and will record computer screens to capture 
user activity. The total length of the experiment, including briefing, completing the censorship 
circumvention tasks, exit survey, and debriefing, will take about an hour. 
Participants will be compensated \$30 for their time, which is more to cover 
minimum wage for an hour and any transportation costs to the lab.  \\

\noindent {\bfseries Experiment Flow} 
We start off the experiment by telling them that they are in an adversarial setting, and that some websites are blocked and some websites are not. We will instruct them to visit a non-blocked website and a blocked website on a ``standard'' browser (one that is not used for censorship circumvention, such as Chrome, Safari, or Firefox) to illustrate the situation.
Then, the participants will be informed of the role they they will be playing in the study. Particularly,
they will be informed that they are a normal citizen in a censorship environment
which prevents them from visiting certain websites, that they are visiting the website for non-critical leisurely purposes,
and that there is a minimal chance of risk. This information is given so that the participants can have an idea of the role they are playing, and to achieve consistency across participants. Their behavior might be different if they were visiting websites to accomplish a critical task, or if they knew that there were great risks. 

After explaining the situation and their role, we will explain what Tor is, and how they can use it if necessary, and instruct them to complete the a set of tasks which will require visiting blocked websites. Ultimately, the participants will need to configure their Tor Browser to circumvent the simulated censorship environment. These tasks have a three-fold purpose of providing participants with feedback (if they did not correctly configure their browser, they cannot reach the website),  motivating the participants to configure the browser correctly, and providing the researchers with an indicator of successful censorship circumvention. While participants are on the computer, their screens will be recorded to capture how participants configure their browser, and to provide additional evidence that they have completed the tasks. The participants will be unaware that their screens are being recorded, to prevent deviations in behavior. 

After completing the browsing tasks, we will administer a survey that 1) measures participant demographics, participant technical fluency, and participant familiarity with Tor and 2) collects qualitative feedback on from participants on their browsing experience, particularly with respect to any parts of the configuration interface which was confusing to them, but also about the browsing experience as a whole.  A rough draft of the survey can be found \href{http://www.surveygizmo.com/collab/2085559/Tor-Usability-Survey}{here}.
 
 The experiment will conclude with a debriefing, which will inform that the participants of their screen capture and obtaining re-consent for the information.  \\

\noindent {\bfseries Censorship Environment Simulation} 
We plan to simulate three censorship environments.
They are informed by our experience with pluggable transports
and knowledge of commonly seen censorship techniques.
They are not meant perfectly to replicate the network environment
in any particular country. Although inspired by reality, these
abstract simulations are intended to require distinct configurations
of the Tor Browser from our participants. 

\begin{itemize} \itemsep1pt \parskip0pt \parsep0pt
\item {\bfseries Mild censorship} 
(Reflective of countries such as {\color{red}?? and ??}.)
Certain domain names are blocked. Reaching these 
domains requires any use of a censorship circumvention 
tool. The default option to ``connect'' to the Tor network 
directly will circumvent this censor. Additional correct
bridge or proxy configuration are optional. 
\item {\bfseries Intermediate censorship} 
(Reflective of countries such as {\color{red}?? and ??}.)
Certain domain names are blocked. Censorship circumvention
tools are blocked. Since Tor is blocked by blocking all public Tor
relay nodes, the default option to ``connect'' to the Tor network
directly will fail. A choice of a hard-coded bridge (see Figure ~\ref{fig:bridges})
or a valid non-public bridge is required to circumvent this censor.  
Additional correct proxy configuration is optional.
\item {\bfseries Comprehensive censorship} 
(Reflective of countries such as China and Syria.)
Certain domain names are blocked. Censorship circumvention tools
are thoroughly blocked. Tor is blocked by blocking all public
Tor relay nodes, and the censor has examined source code to block
all hard-coded bridge relays in the configuration interface. The default option
to ``connect'' to the Tor network directly will fail. Choosing any other bridges than
``meek-amazon,'' ``meek-azure,'' and``meek-google,'' will fail. This is because 
domain-fronting requires censors to block entire CDNs to also block this
transport (which will cause huge collateral blocking damage), making it resistant to agressive censorship environments. Additional details,
see ~\cite{fifield2015blocking}.\\
\end{itemize}

\noindent {\bfseries List of Tasks} 
In our experimental setup, successful completion of 
these given tasks requires of correct configuration.
The difficulty of circumventing  censorship to visit the 
websites required for the tasks will vary on the 
simulated censorship environment. Since all participants will be given the 
same set of tasks, regardless of the censorship environment, 
the difficulty of completing the tasks remains constant assuming
correct configuration. The tasks themselves are not intended to
be challenging.

The tasks were initially inspired by the top Alexa sites, 
an indication of representative and relevant browsing behavior. 
From these, we selected sites which were commonly censored, 
but filtered tasks that would require a participant to reveal private information 
(such as login information) for ethical considerations. These tasks
were further refined after performing a pilot study of the experiment. 
We hope the tasks convey an example of how of a user might 
browse the Internet in a censored environment. 

\begin{itemize} \itemsep1pt \parskip0pt \parsep0pt
\item Google search for the population of Zimbabwe. 
\item On Youtube, find a video playing Bach's ``Ode to Joy.'' 
\item Find the Amazon best-sellers in ``Movies \& TV.''
\item On Yahoo, find the exchange rate of dollars to euros.
\item Find the Wikipedia ``History'' portal's featured article. 
\item On Twitter, find the currently trending topics.
\item On Bing Maps, find directions from Time Square to Carnegie Hall.
\end{itemize}

\section{What we hope to accomplish}

We plan to finish the user study by the end of the semester. 
From this experiment, we hope to accomplish the following: 

\begin{itemize} \itemsep1pt \parskip0pt \parsep0pt
\item {\bfseries Test the Tor configuration interface} We will perform the largest-scale user study 
of Tor to date, measuring how users configure Tor in three different adversarial settings. 
\item {\bfseries Test an improved configuration interface} With the measurements collected
from testing the interface as-is, Nathan will be designing ways to improve the interface to minimize
time taken, paths taken, and error states reached.
\item {\bfseries Test an alternative configuration interface} A Tor developer has offered
to design a mock-up interface for an alternative configuration interface which will
greatly automate the process. There are tradeoffs between ease of use and transparency
of the system's, and ethical considerations with loggable errors resulting from automated configuration. 
\item {\bfseries Push changes} We have the support of Tor developers
to improve this interface. Additionally, since the configuration interface does not require 
any changes to the Tor Browser functionality, improvements to the interface will
be easy to deploy. 
\end{itemize}

\section{Resources}
\noindent Our online artifacts of the work done during this class, Fall 2015,
are below: 
\begin{itemize} \itemsep1pt \parskip0pt \parsep0pt
\item \href{https://github.com/lindanlee/circumvention-ux-tor}{github repo with experiment plans, code, and paper}
\item \href {https://github.com/lindanlee/circumvention-ux-tor/blob/master/pilot/1-strict.mp4}{pilot video 1}
	\href{https://github.com/lindanlee/circumvention-ux-tor/blob/master/pilot/2-lax.mp4}{pilot video 2}
\item \href{https://github.com/lindanlee/circumvention-ux-tor/blob/master/setup/setup-environment}{experimental setup code: firewall and screen capture} 
\item \href{https://github.com/lindanlee/circumvention-ux-tor/blob/master/setup/takedown-environment}{experimental takedown code: saving files and cleanup} 
\end{itemize}

% from anonymity study
Our online artifacts of study of Tor as an anonymity tool, such as 
the summary, results, and resulting browser changes are below:
\begin{itemize} \itemsep1pt \parskip0pt \parsep0pt
\item \href{https://trac.torproject.org/projects/tor/wiki/org/meetings/2015UXsprint}{blog post summary}
\item \href{https://blog.torproject.org/blog/ux-sprint-2015-wrapup}{changes made to Tor}
\item \href{https://people.torproject.org/~dcf/uxsprint2015/}{subtitled screen videos}
\end{itemize}

\bibliographystyle{abbrv}
\bibliography{circumvention-experiment.bib} 
\end{document}
