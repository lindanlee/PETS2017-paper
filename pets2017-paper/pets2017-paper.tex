\documentclass{template}
\usepackage{color}
\usepackage[hyphens]{url}
\usepackage{longtable}
\usepackage{graphicx}
\usepackage{enumitem}
\usepackage{pdfpages}
\usepackage{hyperref}

\def\etal{{\it et al.~}}
\newenvironment{packed_enum}{
\begin{enumerate}
  \setlength{\itemsep}{1pt}
  \setlength{\parskip}{0pt}
  \setlength{\parsep}{0pt}
}{\end{enumerate}}
\newenvironment{packed_item}{
\begin{itemize}
  \setlength{\itemsep}{1pt}
  \setlength{\parskip}{0pt}
  \setlength{\parsep}{0pt}
}{\end{itemize}}

\begin{document}

%what I really want to say is that the interesting aspect of this usable design is that we had to 
%account for people who wouldn't need to use this launcher at all and also for at-risk users.
%... but in one catchy title. I can't think of one.
\title{Designing for Multiple Use and At-Risk Users: Tor Configuration Launcher}
\numberofauthors{1}
\author{
 \alignauthor Linda N. Lee, David Fifield, Nathan Malkin \\
   \vspace{0.5em}
   \affaddr{University of California, Berkeley} \\
   \affaddr{\{lnl,fifield,nmalkin\}@cs.berkeley.edu}\\
}
\maketitle

\begin{abstract}
{\color {red} 
stuff here
}
\end{abstract}

\keywords{User Studies, Tor, Security, Censorship, Anonymity}

\section{Introduction}
{\color {red} 
Introduce what Tor is, and what the configuration interface does in the Tor Browser Bundle. Explain the story of how it began as an anonymity tool, but it has since changed to become a censorship circumvention tool. Give statistics of how many people use it (although we can't say how many for what reason exactly), and interesting trends, such as a spike in use around election times or some big political event. 
}

This study aims to understand of what is confusing about the configuration process and what find changes make the process easier. Our study consists of three stages. User interactions with the current interface in various censorship environments generated the problems we address in the study. User feedback and observations of users steered the design process for an improved configuration interface. User metrics, such as success rate and time to completion, measured how well the interfaces served their purpose. 

{\color {red}
Explain the state of censorship circumvention today, and what governments do to Tor (block it, etc.). Mention the current consequences of people if they try to connect to Tor. Then drive home why it is important for the interface to be usable. And that there have been previous work on Tor Browser, but not any previous work done for its usability as a censorship circumvention tool.
}

Currently, x\% of users fail and with the current interface for reasons a, b, and c. Of those that do succeed, the average time to completion is y. This can lead to a bad user experience, users quitting Tor, or causing high-risk users to make mistakes. With the new design, only x'\% of users fail to configure successfully and the average time to completion is y'. Tor has already implemented some of these changes. 

\section{Background} 
\subsection{The Interface} 
{\color {red} storyboard of the old interface 

\begin{enumerate} \itemsep1pt \parskip0pt \parsep0pt
    \item No bridge, no proxy
    \item with bridge, no proxy
    \item No bridge, with proxy
    \item With bridge, with proxy
\end{enumerate}
}
\subsection{Kinds of Users} 
\subsection{Network Components} 
\subsection{Censorship Environments} 

\section{Goals/Evaluation Criteria} 
\begin{enumerate} \itemsep1pt \parskip0pt \parsep0pt
    \item {\bfseries Task Completion} Almost all can successfully connect to Tor
    \item {\bfseries Time to Completion} Time to complete, east of user for the average user. 
    \item {\bfseries Safe for High-Risk Users} It should be possible to configure Tor with the interface that it doesn't leak that they are using Tor. 
\end{enumerate}

\section{Qualitative Analysis of the Existing Interface (Study 1)}
\subsection{Motivation} 
\subsection{Methodology} 
{\color {red} 
What to talk about in this section: 
\begin{itemize} \itemsep1pt \parskip0pt \parsep0pt 
\item experiment design
\item censorship environments
\item user tasks
\item user screening and recruitment
\item payment
\item no risks to user, irb approved
\end{itemize} 
}

\subsection{Results} 
{\color {red}
We noticed four common challenges:  

\begin{itemize} \itemsep1pt \parskip0pt \parsep0pt
\item {\bfseries Challenge 1:} People don't know how to choose between ``connect directly'' versus ``configure'' on the first screen. As seen in Figure \ref{fig:window1}, there is some guidance for the users, but we found this largely unhelpful since people do not understand the text on the screen. 
\item {\bfseries Challenge 2:} People feel compelled to set up a bridge, even if they don't need one, because they do not know how censorship works. Participants knew that they were censored to some degree since we had simulated censorship of websites. However, bridges are only necessary when the government has taken active steps to censor Tor itself.
\item {\bfseries Challenge 3:} When their first attempt to connect fails, the interface directs people to the last window they saw before the connection failure. The problem is not necessarily where the interface directs users, misguiding them. It was common for the interface to redirect users to the proxy window in Figure \ref{fig:window5}, even when they needed to pick another bridge. 
\item {\bfseries Challenge 4:} When the first attempt to connect fails, people did not know what to do or what to try next. The feedback given to users was not understood (see Figure \ref{fig:error}) and also did not guide users into specific actions (such as trying a different bridge, trying the same configuration again, try connecting directly, etc.).
\end{itemize}
}

\section{Redesigning the Configuration Interface}

{\color {red} 

Only use the following if it helps explain some things. Make sure the tone of things are positive, and explain that making a good interface is harder than it seems.

{\bfseries What it's trying to do}
\begin{enumerate} \itemsep1pt \parskip0pt \parsep0pt
    \item separate people who use Tor for anonymity and for censorship 
    \item if anonymity, to connect people directly
    \item if censorship, guide users to make the correct configuration
    \item minimize network traces to protect high-risk users
\end{enumerate}


{\bfseries Why certain decisions were made}
{\color {red} 
\begin{enumerate} \itemsep1pt \parskip0pt \parsep0pt
    \item {\bfseries Asking the User} asking users about their situation rather than hardcoding or probing
    \item {\bfseries No Explicit Advice} no suggestion of what to try and in what order
    \item {\bfseries No Automation} manual configuration of bridges and proxies required
    \item {\bfseries Optimizing for the Average Case} connect, bridges before proxy
\end{enumerate}
}

What to talk about in this section: 
\begin{itemize} \itemsep1pt \parskip0pt \parsep0pt 
\item Talk about the process of translating observations and feedback into feature changes. 
\item Polishing the interface by working with a designer, using heuristics, and using technical information. 
\item Iterative design and feedback cycle. 
\item How we made a functional, instrumented prototype (forking from the interface repo on github and hacking on it) and instrumented the old prototype. 
\end{itemize} 

Talk about how the design as a conservative design. And why the conservative choices were made. For at-risk users, maintainability, etc.
}

\section{Quantitative Analysis of the Interfaces (Study 2)}

\subsection{Motivation} 

\subsection{Methodology} 
{\color {red} 
What to talk about in this section: 
\begin{itemize} \itemsep1pt \parskip0pt \parsep0pt 
\item experiment design
\item how we chose our metrics
\item how we collected our metrics
\end{itemize} 
}

\subsection{Results} 

\section{Future Work} 
{\color {red}
\noindent {\bfseries Changing the Current Interface} merge, push. 

\noindent{\bfseries Finishing Touches} 
\begin{itemize} \itemsep1pt \parskip0pt \parsep0pt 
\item {\bfseries Detecting the need for proxies.} We hacked this together for now. 
\item {\bfseries Handling rare error cases.} Mention clock drift and proxy connection cases.
\item {\bfseries Additional user feedback.} Animation in the progress window. 
\item {\bfseries Stylization.} Comply with design style guides, if any. 
\end{itemize} 

\noindent{\bfseries Measuring Impact} Tor metrics to see if this is helping at the large scale, and for future user study work in general. 
}

\section{Discussion} 

Additional changes that could be made: 
\begin{itemize} \itemsep1pt \parskip0pt \parsep0pt 
\item {\bfseries Tell people to click connect.} On the first screen, there are no instructions for the user on what to do. There is a description of what each option means, but there is no guidance on what they should do. Ideally, we would people to click connect, and then try the manual configuration if a direct connection fails. But communicating this may put some users at risk. Based on the results, is the benefit greater than the risk?  
\item{\bfseries Hide the proxy screen.} Don't give users the option to configure a proxy unless it has been detected that it would be necessary. The amount of people who use proxies is low, but then ... this would not give users control in the beginning of the configuration process. 
\item{\bfseries Help the at-risk users.} There are people who would benefit from, on the first attempt, connecting with a meek bridge or custom bridge, without trying a vanilla tor connection or the recommended bridge in the dialog. Currently, we guide these people to the correct decision, but after trying the default bridge first. Helping these users to make a connection safely without ever being logged as connecting to Tor would be a benefit, but might be a case of overfitting to these types of users. A majority of users should not use a meek bridge or custom bridge, so it would be unideal to lead all users to this decision.
\end{itemize} 

Alternative approaches to interacting with the user: 
\begin{itemize} \itemsep1pt \parskip0pt \parsep0pt 
\item{\bfseries Automate the entire configuration process.} Seriously, this can help load balancing, and all sorts of technicalities, and wouldn't harm most of our users. Most users do not configure that well, anyway and would make the same mistakes. 
\item {\bfseries Auto-configure after connect.} After a person has already clicked connect and the connection was unsuccessful, they have already been logged. I am assuming that the significant difference is between being logged trying to connect to Tor or not at all, rather than the number of connection attempts made. This would greatly save our users a lot of headache. 
\item{\bfseries Ask about the risk.} Rather than having the configuration dialog be manual by default, just ask if the users are at risk if the process was automated. If they are not at risk, we can do it automatically, and the at-risk users can configure manually. The issue with this is that people may not answer this question honestly, or they might not know the correct answer to this question. 
\item{\bfseries Ask if they are qualified to make decisions.} Asking users if they know which bridge to choose, and choosing for them if they say they don't know. We probably know better than they do. But there are ethical implications of choosing bridges for them rather them making the mistakes themselves. The issue with this is that people may not answer honestly. 
\end{itemize}


\section{Related Work} 

\section {Acknowledgments}

\section{Conclusion} 

\bibliographystyle{abbrv}
\bibliography{pets-2017-paper.bib} 

\appendix
\section{TODO}
{\color {red} 
Include: 
\begin{itemize} \itemsep1pt \parskip0pt \parsep0pt 
\item participant worksheet
\item participant survey
\item participant recruitment prompt
\item any instructions we gave participants
\end{itemize} 
}


\end{document}
