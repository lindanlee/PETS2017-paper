\documentclass{template}
\usepackage{color}
\usepackage[hyphens]{url}
\usepackage{longtable}
\usepackage{graphicx}
\usepackage{enumitem}
\usepackage{pdfpages}
\usepackage{hyperref}

\def\etal{{\it et al.~}}
\newenvironment{packed_enum}{
\begin{enumerate}
  \setlength{\itemsep}{1pt}
  \setlength{\parskip}{0pt}
  \setlength{\parsep}{0pt}
}{\end{enumerate}}
\newenvironment{packed_item}{
\begin{itemize}
  \setlength{\itemsep}{1pt}
  \setlength{\parskip}{0pt}
  \setlength{\parsep}{0pt}
}{\end{itemize}}

\begin{document}

\title{An Inclusive, Usable Redesign of the Tor Configuration Launcher}
\numberofauthors{1}
\author{
 \alignauthor Linda N. Lee, David Fifield, Nathan Malkin \\
   \vspace{0.5em}
   \affaddr{University of California, Berkeley} \\
   \affaddr{\{lnl,fifield,nmalkin\}@cs.berkeley.edu}\\
}
\maketitle

\begin{abstract}

\end{abstract}

\keywords{User Studies, Tor, Security, Censorship, Anonymity}

\section{Introduction}
We need to design for at least three types of users: users who use Tor for anonymity, users who use Tor for censorship circumvention who are not at risk, users who use Tor for censorship circumvention who are at risk. 

What makes this design unique is that we have to handle two use cases (anonymity and censorship circumvention) and that there are at-risk users we need to consider. We can't lean toward helping them configure their connection all the time because a lot of people use Tor for anonymity or casual browsing, and to do this would be over-fitting to the censorship evasion population (and technically, is a burden on bridges). We also cannot automate anything, unless we want to put users at risk. Also, for maintainability, we would ideally love it if users could just choose their bridge settings, since any automation or device is bound to be different from country to country and also will become out of date sooner or later. 

\section{The Configuration Interface}

\section{Methodology}

\subsection{Evaluation}
%Some references to keep in mind when explaining:
%\href{why 5 users}{http://www.nngroup.com/articles/how-many-test-users/}
%\href{screening}{http://www.userfocus.co.uk/articles/screeners.html}
%\href{summative and usability mixed model}{http://www.usabilitybok.org/summative-usability-testing}

\subsection{Design}

\subsection{Testing}
{\color {red} TODO}

\subsection{Logistics} 
{\bfseries Simulated censorship environments}
{\bfseries Online Tasks}

\section{Results} 

\subsection{Evaluation} 
\subsection{Design} 
\subsection{Validation} 
\subsection{Recommendations} 
TODO, write about what worked, what didn't, and what we recommend.

\section{Future Work}
In addition to merging our code with the source code: 

\begin{itemize} \itemsep1pt \parskip0pt \parsep0pt 
\item {\bfseries Detecting the need for proxies.} We hacked this together for now. 
\item {\bfseries Handling rare error cases.} Mention clock drift and proxy connection cases.
\item {\bfseries Additional user feedback.} Animation in the progress window. 
\end{itemize} 

\section{Discussion}
Talk about how the design was a conservative version. And why the conservative choices were made. For at-risk users, maintainability, etc.

Additional potential changes that could be made to the browser: 
\begin{itemize} \itemsep1pt \parskip0pt \parsep0pt 
\item {\bfseries Tell people to click connect.} On the first screen, there are no instructions for the user on what to do. There is a description of what each option means, but there is no guidance on what they should do. Ideally, we would people to click connect, and then try the manual configuration if a direct connection fails. But communicating this may put some users at risk. I would argue that the benefit is higher than the risk. 
\item {\bfseries Auto-configure after connect.} After a person has already clicked connect and the connection was unsuccessful, they have already been logged. I am assuming that the significant difference is between being logged trying to connect to Tor or not at all, rather than the number of connection attempts made. This would greatly save our users a lot of headache. 
\item{\bfseries Hide the proxy screen.} Don't give users the option to configure a proxy unless it has been detected that it would be necessary. The amount of people who use proxies is low, but then ... this would not give users control in the beginning of the configuration process. 
\item{\bfseries Ask about the risk.} Rather than having the configuration dialog be manual by default, just ask if the users are at risk if the process was automated. If they are not at risk, we can do it automatically, and the at-risk users can configure manually. The issue with this is that people may not answer this question honestly, or they might not know the correct answer to this question. 
\item{\bfseries Ask if they are qualified to make decisions.} Asking users if they know which bridge to choose, and choosing for them if they say they don't know. We probably know better than they do. But there are ethical implications of choosing bridges for them rather them making the mistakes themselves. The issue with this is that people may not answer honestly. 
\end{itemize}

\section{Resources}
{\color {red} appropriate links here.}

\bibliographystyle{abbrv}
\bibliography{pets-2017-paper.bib} 

\appendix
\section*{Old Configuration Interface}
\section*{New Configuration Interface}


\end{document}
