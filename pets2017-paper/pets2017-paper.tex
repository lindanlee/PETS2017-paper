\documentclass[USenglish,oneside,twocolumn]{article}
\usepackage{color}
\usepackage[hyphens]{url}
\usepackage{longtable}
\usepackage{graphicx}
\usepackage{enumitem}
\usepackage{pdfpages}
%\usepackage{hyperref}

\usepackage[utf8]{inputenc}%(only for the pdftex engine)
%\RequirePackage[no-math]{fontspec}%(only for the luatex or the xetex engine)
\usepackage[big]{dgruyter_NEW}
 
\DOI{foobar}

\cclogo{\includegraphics{by-nc-nd.pdf}}
  
\begin{document}
 
  \author*[1]{Linda Lee}

  \author[2]{David Fifield}

  \author[3]{Nathan Malkin}

  \author[4]{Ganesh Iyer}

  \author[5]{Serge Egelman}
  
  \author[6]{David Wagner}

  \affil[1]{University of California Berkeley, E-mail: \mbox{lnl@cs.berkeley.edu}}

  \affil[2]{University of California Berkeley, E-mail: \mbox{fifield@cs.berkeley.edu}}

  \affil[3]{University of California Berkeley, E-mail: \mbox{nmalkin@cs.berkeley.edu}}

  \affil[4]{University of California Berkeley, E-mail: \mbox{ganesh.v@berkeley.edu}}
  
  \affil[5]{University of California Berkeley and International Computer Science Institute, E-mail: \mbox{egelman@cs.berkeley.edu}}
   
  \affil[6]{University of California Berkeley, E-mail: \mbox{daw@cs.berkeley.edu}}

  \title{\huge Tor's Usability for Censorship Circumvention}
  %Internet Freedom Made Easy: On Improving Tor’s Usability for Censorship Circumvention
  %Towards Usable Censorship Circumvention

  \runningtitle{Tor's Usability for Censorship Circumvention}

  %\subtitle{...}

  \begin{abstract}
{Tor has grown beyond its original purpose as an anonymity tool and has 
since become an important censorship circumvention tool. It is now listed
as one of the ways that normal people use Tor.
We specifically examine its usability as a censorship circumvention tool,
an essential facet for adoption and use. We focus our analysis on the connection 
configuration interface of Tor Browser. We find that the Tor Browser 5.0.3. connection
configuration interface has a success rate of {\color {red} TODO}\% with the 
successful users having an average completion time of {\color {red} TODO} minutes. 
In this paper, we explore common challenges users faced during the configuration
process and redesign the interface to improve success rate to {\color {red} TODO}\% 
and average completion time to {\color {red} TODO} minutes.}
\end{abstract}
  \keywords{User Studies, Tor, Security, Censorship, Anonymity}
%  \classification[PACS]{}
 % \communicated{...}
 % \dedication{...}

  \journalname{Proceedings on Privacy Enhancing Technologies}
\DOI{Editor to enter DOI}
  \startpage{1}
  \received{..}
  \revised{..}
  \accepted{..}

  \journalyear{2015}
  \journalvolume{2015}
  \journalissue{2}

\maketitle

\section{Introduction}

Although Tor~\cite{dingledine2004tor} was not designed to be a censorship circumvention tool, users started
using it as one, since Tor's role as a proxy to anonymize
people also worked to circumvent censorship. Now, circumventing censorship is a
common use case for Tor, and current censorship circumvention methods are
incorporated into Tor, making it a viable censorship circumvention tool even in the face
of adversaries. In fact, some countries unsuccessfully attempt to block Tor for this reason~\cite{winter2012great}. 
To our knowledge, this is the first user study investigating the usability of Tor as a 
censorship circumvention tool.

We believe that this is a great opportunity for user research. We have the opportunity to help users circumvent censorship more easily and provide them with extra security features, such as anonymity. Not only does improving the interface help the user trying to circumvent censorship, it helps other users who use Tor as an anonymity system by increasing the number of overall users~\cite{dingledine2006anonymity}. 

This study aims to understand of what is confusing about the configuration process and what find changes make the process easier. Our study consists of three stages. User interactions with the current interface in various censorship environments generated the problems we address in the study. User feedback and observations of users steered the design process for an improved configuration interface. User metrics, such as success rate and time to completion, measured how well the interfaces served their purpose.  

Currently, {\color {red} TODO}\% of users fail and with the current interface for reasons {\color {red} TODO}, {\color {red} TODO}, and {\color {red} TODO}. Of those that do succeed, the average time to completion is {\color {red} TODO}. This can lead to a bad user experience, users quitting Tor, or causing high-risk users to make mistakes. With the new design, only {\color {red} TODO}\% of users fail to configure successfully and the average time to completion is {\color {red} TODO}. Tor has already implemented some of these changes.

The qualitative user study gives insight into possible issues with
the deployed interface. 
The redesign process was guided by the participant observations, 
heuristic evaluations and empirical goals. 
Quantitative user studies provide empirical evaluations of the deployed
and redesigned interface. 

An exploration of the problems with the existing Tor Browser 
configuration interface through user observation and interviews can be 
found in Section~{\color{red} TODO}. A heuristic evaluation of the existing interface and new interface is in Section~{\color{red} TODO}. We empirically show 
that the redesigned interface improves completion rate, reduces time to completion, 
and provides a safe path for at-risk users in Section~{\color{red} TODO}. 

\section{Related Work} 
There have been three published user studies on Tor. Clark et. al examined various deployment
options for Tor Browser, such as Vidalia, Privoxy, Torbutton, and Foxyproxy, and found that none of them 
were satisfactory from a usability perspective~\cite{clark2007usability}. Fabian et. al show that Tor's added latency 
~\cite{dingledine2009performance} causes users
to be frustrated, cancel requests more often, and prevents user adoption~\cite{fabian2010privately}. 
Norcie et. al found found that 
64\% of users are unable to continue with installation or browsing at least once due to difficulties~\cite{norcie2012eliminating}. 

There have been been no published usability evaluations of
Tor Browser since the 4.0 series, which introduced radical UI changes. 
The most recent usability effort is an unpublished pilot study  by Lee and Fifield~\cite {uxsprint} 
that tests the download, install, and  user interface of Tor Browser.  This study uncovered a number of issues~\cite{uxsprint2015-tickets}. Changes made are reflected in series 5.1 and later. 

Previous user studies have not focused on specific features in isolation, while we choose to focus our study on 
the configuration interface. The configuration interface guides users through setting up network components required to circumvent censorship. 
This study examines that interface, and effectively, the usability of Tor as a censorship circumvention tool. 

\section{Background} 
This section examines Internet censorship, interface network components that circumvent censorship, and the Tor Browser 5.0.3 configuration interface. 

\subsection{Censorship Today} 
Internet censorship is widespread throughout the world~\cite{faris2008measuring}. Although there are corporations, parental controls, and other forms of censorship, the most common form of censorship is done by a nation-state power, usually without the consent of the people in that residing nation. The risks associated with circumventing Internet censorship are largely based on geographic location, prevalence of censorship circumvention, and political status. 

\subsection{Interacting Network Components} 
Bridges, pluggable transports, and proxies are network components that 
circumvent censorship and interact with the Tor configuration interface. 
These components are shown in Fig.~\ref{fig:topology}.
Bridges and pluggable transports are completely Tor-specific concepts, whereas a proxy is not.  
This information is necessary to configure connections to Tor in censored environments. 

\begin{figure}
\centering
% \includegraphics[width=0.5\textwidth]{topology.png}
{\color {red} TODO}
\caption{
The chain of components involved in connecting to a website over Tor.
Most users do not need a proxy;
similarly only those users who face a censor need a bridge.
In the diagram, ``Tor'' represents all three anonymizing hops through the Tor network.
We have shown the bridge as separate component
because of the special role it plays.
When a bridge is used, it takes the place of the first Tor hop.
}
\label{fig:topology}
\end{figure}

Bridges are unlisted Tor relays that make it possible for a user to connect
to the Tor network even if a censor blocks all publicly listed Tor relays. 
Censors in countries which blocked Tor generally were motivated enough to aggressively
block Tor in other ways, such as traffic analysis. For this reason, a bridge is usually associated
with a pluggable transport that circumvents deep packet inspection. Transports can
assist by obfuscating traffic or reflecting traffic through a content distribution network. 
Proxies help bypass local network restrictions and are typically only
needed in certain cases, such as corporate environments.

\subsection{The Current Configuration Interface} 
Tor Browser provides a configuration interface to guide users through setting up 
their connection. While most can connect to Tor using a direct connection, users in certain censorship environments must
make use of additional settings, such as bridges, proxies, or both. As bridges and proxies can be configured independently or together, there are four ways to set up a connection:\\

\begin{enumerate}
    \item With a bridge and proxy
    \item With a bridge but no proxy
    \item Without a bridge but with a proxy
    \item Without either a bridge or proxy
\end{enumerate}

A series of questions determines which configuration a user requires. The interface asks users questions along the way throughout the configuration to select components they would need to configure. For a successful configuration, users are required to know (or find out) the following:\\

\begin{itemize}
	\item{Whether they are censored} 
	\item{If the Tor Network is censored by their ISP}
	\item{Which bridges work in the current environment} 
	\item{If no bridges work, how to get a custom bridge} 
	\item{If a proxy is required to access the Internet}
	\item{If a proxy is required, the proxy settings}
\end{itemize}

This puts the burden of knowing technical information and configuring correctly on the user. Note that bridges, pluggable transports, and proxies are technical concepts that are likely unfamiliar to the average user. Although burdensome, the interface takes the approach of guiding the user through this technical process in order to give users their own agency in configuring their connection. There has been no previous work to explore the option of automation, but a manual configuration was preferred by Tor developers since it allows the user can have control over the process. If a user had all the relevant information, they would be able to connect without obvious network traces of attempting to connect to Tor. 

\begin{figure}
\small
\begin{itemize}
\renewcommand{\labelitemi}{$\circ$}
\item Connect with provided bridges\\
Transport type:
\begin{itemize}
\item flashproxy
\item fte
\item fte-ipv6
\item meek-amazon
\item meek-azure
\item meek-google
\item obfs3 (recommended)
\item obfs4
\item scramblesuit
\end{itemize}
\item Enter custom bridges\qquad\fbox{Help}\\
\end{itemize}
\caption{
Bridge selection options in Tor Browser~5.0.3.
The obscure strings under ``Connect with provided bridges'' denote various
built-in circumvention technologies (``pluggable transports'').
Under ``Enter custom bridges'', there is a space to paste in
a bridge specification that was obtained out of band.
The ``Help'' button displays instructions on obtaining such
bridge specifications.
}
\label{fig:bridge-options}
\end{figure}

Fig.~\ref{fig:bridge-options} shows the bridge options
presented to a user in Tor Browser~5.0.3.

\begin{figure*}[t]
	\centering
		\includegraphics[width=1.0\textwidth]{old-flow.pdf} 
		\caption{old flow} 
	\label{fig:old-flow}
\end{figure*} 

\section{Goals/Evaluation Criteria}
\label{sec:goals}
We evaluate the existing and redesigned Tor Browser configuration interface 
through the empirical evaluation and heuristic evaluation criteria below. 

A heuristic evaluation is a rule-based evaluation based on time-tested standards
in the user interface design community. We draw from Jacob Nielsen's ten heuristics~\cite{nielsen1994heuristic}
and other industry standards to form the following heuristic evaluation criteria: \\

\begin{enumerate}
    \item  {\bfseries Visibility of system status}: provide feedback on what the system is doing. 
    \item  {\bfseries Error management}: expressing errors in plain language and offering a solution. 
    \item  {\bfseries Understandable language}: explaining the task, rather than technical concepts. 
    \item  {\bfseries Simple design}: showing options on an as-needed basis. 
\end{enumerate}

Our empirical evaluation criteria for the Tor Browser configuration
interface are from common metrics that measure ease of use with a special 
consideration for high-risk users: \\

\begin{enumerate}
    \item {\bfseries Task Completion}: Almost all can successfully connect to Tor
    \item {\bfseries  Time to Completion}: Time to completion. 
    \item {\bfseries Safe for High-Risk Users}: It should be possible to configure Tor with the interface that it doesn't leak that they are using Tor. 
\end{enumerate}

\section{Qualitative Analysis of the Existing Interface (Study 1)}
\label{sec:qualitative}

We performed qualitative research as an exploratory effort to gain an 
understanding of underlying problems, such as reasons for confusion 
during the configuration process and motivations for particular choices in 
configuration. We did this through observing each participant as they 
interacted with Tor Browser 5.0.3 interface and interviewed them about
their experience. This study provides insights into particular problems with 
the current Tor Browser configuration interface, and helps us to design
a new interface which addresses these problems and gives foundation
for hypotheses to test in quantitative research. 

\subsection{Inspection}
A combination of usability inspection methods~\cite{nielsen1994usability}
were used to prepare for the user study. Two researchers conducted a pluralistic 
walkthrough and stepped through various censorship
scenarios, discussing which elements would be involved for that use case, and walking 
through the configuration process. After compiling all the possible paths through the 
interface, feature inspection was performed to list sequences
of features used to accomplish typical tasks, taking note of long sequences or cumbersome
steps. To focus our observations, a heuristic evaluation was performed to mark design issues 
that may cause confusion for users during our study. 

\subsection{Setup}
\label{sec:setup}
We simulated three censorship environments for our qualitative and quantitative user studies. \\

\begin{itemize} \itemsep1pt \parskip0pt \parsep0pt
\item {\bfseries Mild censorship (E1)}: 
(Representative of countries such as France and Australia.)
Certain domains are blocked. Reaching these 
domains requires a censorship circumvention 
tool like Tor, but Tor is not explicitly blocked. 
The default option to ``connect'' to the Tor network 
directly will circumvent this censor. Additional correct
bridge or proxy configurations are optional. 

\item {\bfseries Intermediate censorship (E2)}: 
(Representative of countries such as Tunisia.)
Certain domains are blocked. Censorship circumvention
tools such as Tor are blocked. Since all public Tor
relay nodes are blocked, the default option to ``connect'' to the Tor network
directly will fail. Any choice of a hard-coded bridge
or a valid non-public bridge will circumvent this censor.  
Additional correct proxy configuration is optional.

\item {\bfseries Comprehensive censorship (E3)}:
(Representative of countries such as China and Syria.)
Certain domains are blocked. Censorship circumvention tools
are thoroughly blocked. Tor is blocked by blocking all public
Tor relay nodes, and the censor has examined source code to block
all hard-coded bridge relays in the configuration interface. The default option
to ``connect'' to the Tor network directly will fail. Most bridges will fail,
but ``meek-amazon,'' ``meek-azure,'' and ``meek-google'' still work.
This is because domain-fronting requires censors to block entire CDNs to also
block this transport, which causes huge collateral blocking damage that
makes the censorship very visible and inconvenient. Currently, even
aggressive censorship environments resistant blocking entire CDNs.
(See~\cite{fifield2015blocking} for additional details.)
\end{itemize}

\begin{table}
\centering
\begin{tabular}{r c c c}
& E1 & E2 & E3 \\
% \noalign{\hrule}
websites blocked & X & X & X \\
public relays blocked & & X & X \\
default bridges blocked & & & X \\
\end{tabular}
\caption{
Summary of our simulated censorship environments.
E1 only requires participants to click ``Connect'';
E2 requires selection of an included bridge;
and E3 requires selection of a specific included bridge,
or manual configuration of a custom bridge.
E2's blocking is a superset of E1's;
similarly E3's is a superset of E2's.
}
\label{tab:environments}
\end{table}

These reproducible, stable environments are informed by our experience 
with pluggable transports and knowledge of commonly seen censorship 
techniques. Their goal in our user study is not to replicate the network 
environment in any particular country, but to require our distinct configurations
so that interface paths are tested.

\subsection{Recruitment}
Using established practices from the field of user experience research~\cite{howmanyusers},
we recruited 5 participants for each censorship environment.
We pre-screened~\cite{screening} our participants for diversity of gender, age, technical expertise,
and self-reported familiarity with Tor in each simluated censorship environment for our summative
usability test~\cite{summative}. 

We recruited our users from Craigslist. The recruitment text can be found in 
Appendix~\ref{qualitative-recruitment}. The recruitment posting contained a 
SurveyGizmo online survey that collected information about our participants.
The complete prescreening survey can be found in Appendix~\ref{qualitative-prescreening}.  

We chose our participants based on the pre-screening information to have 
at least one person who has never heard of Tor, at least one person who has 
only heard of Tor, and exactly one person who has previously used Tor in each
environment. We also tried to evenly distribute any participants who had technical
expertise or used particular security tools throughout the censorship environments. 

Out of our 16 participants, ages ranged from 20 to 62
(mu= 24.5, sigma = 12.6). 53.3\% were male. 
93.3\% of our participants had at least
a college education. 5 had previously heard of Tor but not used it, while 
and 5 had used Tor previously. 


\subsection{Procedure}
We conducted a qualitative evaluation of the interface through real-time observations of 
users interacting with the configuration interface in a simulated censored environment.
There was no researcher-user interaction during the experiment. A follow-up interview 
detailed their thoughts about the configuration process.

The one-hour, single-participant procedure begins when a participant enters a small 
room with a single computer, which is equipped with Tor, Chrome, Firefox, Internet Explorer, 
and VLC (for screen recording). A participant is firstly informed of 
the risks of the study and consenting to data collection. If they consent, the 
experiment begins. A researcher reads a script (Appendix~\ref{qualitative-script}) that 
informs the participant of the simulated censorship environment and
instruct them to visit a sample blocked website and a sample non-blocked website on a
non-Tor browser of their choice to illustrate the situation. Then, participants are asked to 
complete a worksheet (Appendix~\ref{participant-worksheet}) that gives information
on their censorship environment and instructs them
to visit one blocked website and one non-blocked website.
 
We chose Wikipedia's featured article of the day as the blocked website and 
the CNN homepage as our non-blocked website because the familiarity 
that most users have with these websites makes the browsing task relatively easy, 
which focuses participants' attention to configuring Tor Browser. 
After instructions, researchers stepped out of the room so that there was no interaction
between the participant and researcher for the rest of the session. Participants' screens 
were recorded and streamed to another room, where the researchers were able to 
observe how a participant configured their browser. Participants had an 
average of 45 minutes to complete their worksheet. 

At this point, the participants do not know the details of their censorship environment,
only that they are actively being censored. Ultimately, participants needed to configure Tor Browser to 
circumvent the simulated censorship. 

After users completed the browsing tasks or have spent the rest of the time
trying to configure Tor Browser, we interviewed participants about their experience.
We performed a live transcription of the interview to avoid recording voices, which is
considered personally identifying information. 
We asked three standard questions asking about their general experience, 
confusing interface features, and soliciting feedback for improvements. We followed up
with specific questions we had for a particular participant from observing their screen. 
This was to verify any hypothesis we had about the participant (i.e. ``they didn't know what to do on window~3'').  
After their interview, participants were informed that the experiment was over and 
given their payment of~\$30 for their time. 

\subsection{Results} 
In this section, we discuss problems encountered by our participants at each step of the configuration process and solutions to those problems. We conclude with our goals for improving the configuration interface. Because of the small sample size of qualitative studies, we can only state that these are problems that participants can encounter when configuring. All quotes are not verbatim, but from best-effort live transcriptions of participant interviews.

\subsubsection{User Communication} 
Many participants, including participants who were pre-screened as having high technical ability and previous experience with using Tor, were not familiar with the vocabulary. 

\begin{figure}[t]
  \centering
    \includegraphics[width=0.5\textwidth]{error.png}
    \caption{An example of a technical error message which our participants did not understand.}
\label{fig:error}
\end{figure}

\begin{quotation}
\noindent P2:\textit{"I don't know what any of those means (list of bridges),or what that (proxy)
 means at all."}\\
 
 \noindent P3:\textit{``The vocabulary is really challenging, for someone not doing IT work.''}\\
 
 \noindent P4:\textit{``I didn't know why I was getting back `establishing encrypted directory'--what does that mean?''}\\
\end{quotation} 
The text in the interface intended to guide users through the process was not understood by our participants. The text also gave our participants a sense that the process intimidating for people who did not have a technical background. 

\subsubsection{Connect vs. Configure} 
The first task for our participants was to decide between connecting directly to the Tor network or configuring their connection. Text in the interface tells people that connecting directly ``will work in most situations'' and instructs to configure a bridge or proxy if the ``Internet connection is censored or proxied.'' We found that a majority of our participants did not take the time to read the text on the screen and that if they did, the text did not always soundly influence their decision.

Participants across all censorship environments connected directly to the Tor network. When they did, they did so because they were taking the path of least resistance first or intimidated by the configuration process. 
\begin{quotation}
\noindent P9: \textit{``Configure seemed manual, so I clicked connect.''}\\

\noindent P14:\textit{``The words were confusing. I don't really understand computers. I don't know what configure means in this setting. Hmm, is it going to crash? When you see connect, you want to click it because you really want it to connect.''} 
\end{quotation}

Bridges are only necessary when Tor relays are censored, but participants are not able to determine when this happens, or what the difference between a blocked website and a blocked relay. Participant 10 and several other participants in the mild censorship environment opted to configure a connection when a direct connection to the Tor network would have worked. 

\begin{quotation}
\noindent P10: \textit{``I was censored, so I picked the configure.''}
\end{quotation}

Users do not make the optimal choice between connect and configure. This leads to discouragement from attempting a failed connection and wasted time spent waiting for feedback, which lessen the chance for a successful connection. 

\subsubsection{Bridge Configuration} 
When a participant chooses to configure their connection, their next step is to determine whether they need to configure a bridge, and to configure one if necessary. Answering yes to ``Does your Internet Service Provider (ISP) block or otherwise censor connections to the Tor Network'' directs to a bridge configuration screen, whereas answering no skips that screen. We observed that multiple participants struggled with the technical language on the screen, and ultimately decided to try process of elimination. Most participants went with the default option to not configure a bridge. 

\begin{quotation}
\noindent P7:\textit{``The bridge screen was the most challenging. It seemed very technical to me. I don't know what a bridge is.''}\\

\noindent P12:\textit{``I decided which options to choose by process of elimination with trial and error.''}
\end{quotation} 

On the bridge configuration screen, participants need to choose between configuring a built-in bridge versus a custom bridge. Participants were unsure of the difference between a built-in bridge and a custom bridge and also why one would use a custom bridge. Most chose obfs3 as their transport, because those were the default options on the screen. 

\begin{quotation} 
\noindent P8:\textit{``I have no clue what's the difference between flashpoxy, fte, etc. I need to know why the built-in ones aren't working. And why do I need a custom bridge if there are options built in?'}\\

\noindent P7:\textit{``Since it (obfs3) said recommended, it helped actually, and I selected it because it was chosen. I saw the custom bridges option, but I didn't know what to enter there so I went with this (obfs3).''}
\end{quotation} 

Participants in the advanced censorship environment were required to select a meek bridge (not the default) or to configure a custom bridge. After the first pluggable transport failed and even if the participants deduced that the source of the failed connection was a mis-configured bridge, the pluggable transport names only confused participants and caused odd behaviors. Participants chose transports at random, in order of the list starting from the default, in order of the list starting from the top, choosing familiar-sounding transports first (i.e. meek-amazon over fte), and choosing technical-sounding transports first (i.e. fte over meek-amazon). 

\begin{quotation}
\noindent P4:\textit{``When I saw obfs3 as the recommended option, the next option logically for me was obfs4.''}\\

\noindent P5:\textit{``I tried the pluggable transport that used google, because I noticed that google was working (when I was using a non-Tor browser).''}
\end{quotation} 

Users are not qualified to choose transports in an educated order. When our participants chose incorrectly, they spent time testing their wrong hypothesis and got discouraged with each wrong decision. Our participants were resilient and tried for 40 minutes, but we believe that this is a byproduct of the experimental setting and compensation.

\subsubsection{Proxy Configuration} 
After the bridge configuration, the participant is asked if they need a proxy and if so, to configure one. This question also was too technical for most participants.

\begin{quotation}
\noindent P11:\textit{``I think you can only answer this question (does this computer need a local proxy) if you know what a local proxy is. Otherwise, you have no chance..''}
\end{quotation}

For our participants, trying to unsuccessfully configure a proxy was the only mistake that resulted in failure. On the first attempt at configuration, most chose to not configure a proxy. Since the interface passively re-displayed a proxy-related window after an attempted connection regardless of the error, participants incorrectly assumed that they needed a proxy. Participants did not check their incorrect assumption~\cite{wason1960failure}. All who chose to configure a proxy were unsuccessful. 

\begin{quotation}
\noindent P15:\textit{``I didn't know if this computer had any proxy information. I wasn't able to find it if it did.''}
\end{quotation}

None of our censorship environments required our participants to configure a proxy for a successful connection. Participants followed the directions to look at the Internet settings, but they never found the (nonexistent) information.

\subsubsection{Progress Bar} 
Users were generally displeased with the lack of feedback on the progress bar. Users do tolerate delays if they are for security reasons, but only if they understand the reason~\cite{egelmanplease}. Participants in our study usually did not understand the security reason, and experienced delays up to 2 minutes, even in the ideal case in which a user chose the correct configuration on the first attempt. If users chose particular wrong configurations (i.e. a syntactically valid but nonfunctional proxy), they would be waiting for an indefinite period of time. 

\begin{quotation}
\noindent P16:\textit{``There doesn't seem to be a timeout on any of this stuff. Am I waiting long enough? It should work immediately.''}\\

\noindent P6:\textit{``How long does this actually take (to connect)? This took way too long, and I didn't know how long it should take.''}

\noindent P1:\textit{``It was hard to figure out if the progress bar wasn't moving because the connection was censored, or if it was just slow.''}
\end{quotation}

Additionally, the progress bar remains empty on subsequent attempts to connect to the Tor Network until the percent progress supersedes the last displayed progress value (i.e. a participant who saw 30\% progress on their first attempt would see 0\% progress in the progress bar until subsequent attempts got at least 40\% progress). The negative feedback of a 0\% progress bar would cause users to assume that their subsequent attempts were wrong, even if they were correct. 

\section{Redesigning the Configuration Interface}
\begin{figure*}[t]
	\centering
		\includegraphics[width=1.0\textwidth]{new-flow.pdf} 
		\caption{new flow} 
\end{figure*} 

\begin{figure*}[t]
	\centering
		\includegraphics[width=1.0\textwidth]{bridge-screens.pdf} 
		\caption{bridge screens} 
\end{figure*} 

\begin{figure*}[t]
	\centering
		\includegraphics[width=1.0\textwidth]{proxy-screens.pdf} 
		\caption{proxy-screens} 
\end{figure*} 

Tor Browser 5.0.3. configuration interface does not adequately meet our heuristic evaluation criteria from Section~\ref{sec:goals}: visibility of system status, user control and freedom, error management, understandable language, and minimal design. We rate the severity of the issue by its frequency, impact, and persistence~\cite{nielsen1994heuristic}:\\

\begin{itemize}
\item {\bfseries Major Problem}: Error messages do not express errors in plain language nor offer solutions. 
\item {\bfseries Major Problem}: The text was too technical for an average user to understand. 
\item {\bfseries Minor Problem}: The configuration is not visible to the users before they connect or while they are connecting.
\item {\bfseries Cosmetic Problem}: The interface can be simplified visually.  
 \end{itemize} 

To fix the aforementioned problems, we made the following changes: \\

\begin{enumerate}
\item {\bfseries Added instructions on what to try next on errors.} When an error occurs, text advice on what to try next is shown to the user. Such advice includes trying the connection again, choosing a different bridge, or trying a connection without a proxy. 
\item {\bfseries Reduced overall amount of text and made text less technical.} We make the text more task-centric by focusing on instructing users through the configuration process. Since users generally couldn't understand the technical concepts enough to influence their decisions, giving direct guidance may be a better option. 
\item {\bfseries Added system status visibility.} Before any attempt to connect to the Tor network, a summary screen displays if a bridge or proxy is configured, and if so, what the configurations are. The progress screen also displays the configurations while the connection attempt is being made. 
\item {\bfseries Reduced visual clutter.} We showed options on an as-needed basis and we minimized the amount of text on the screen. For instance, additional fields were hidden until the relevant options were chosen. Additional stylistic simplifications were made, such as removing grouping boxes. 
\end{enumerate} 

Other changes improve empirical goals from Section~\ref{sec:goals}: task completion, time to completion, and safe for high-risk users. 5-6 improve task completion, 7-8 reduce time to completion, and 9-10 builds users' mental models of censorship circumvention and network components involved. Users' full control of the configuration process remains preserved.\\

\begin{enumerate}
\setcounter{enumi}{4}
\item {\bfseries Added guidance on choosing connect vs configure.} We labeled the configure option as advanced, manual, and only for heavily censored situations. 
\item {\bfseries Added explicit advice on choosing bridge transports.} The default bridge is still obfs3. There is added text that advises users to try a meek bridge if obfs3 does not work. 

\item {\bfseries Eliminated technical questions.}  We removed questions that determined whether a bridge and proxy should be configured, which were highly technical and challenging for users to answer. There are two fewer screens in the interface.
\item {\bfseries Added auto-detect for proxies.} Auto-detection of proxies can be done locally, by scanning system files. This is safe even for at-risk users, and greatly simplifies the configuration process. 

\item {\bfseries Switched ordering to configure proxies first.} To build users' mental models, network components are configured in a topologically sequential order. Previously, proxies were put after bridge configuration because only a small fraction of users require proxies. With auto-detection, configuring a proxy before a bridge will not burden the users. 
\item {\bfseries Added clear feedback to the progress bar.} We switched the continuous progress bar to a discrete checkpoint based progress bar that shows network components involved in connecting to the Tor network. This aligns with the mental model we are trying to build. As connections are made with components, the progress bar displays a green check, giving immediate feedback to the user. For instance, a user can can now see that the bridge was configured correctly, and they just need to wait to connect to the Tor network. Upon failure, users can see which components have succeeded and which have failed.
\end{enumerate}

The redesigned interface conservatively interacts with users by automating only safe configuration processes and still allowing users to have control over configuring all interacting network components. The design requires more maintenance because of the explicit advice given on which order to select bridges and advice on error messages, which may change as the censorship environments change. 

\section{Quantitative Analyses of the Interfaces (Study 2)}
\label{sec:quantitative}
We performed quantitative research to quantify the existing problems
and the impact of our redesign. Specifically, we collected all events, transitions, 
and state for Tor Browser 5.0.3 and the redesigned Tor Browser.

\subsection{Methodology} 
This section shows the details of our quantitative user study, such as the censorship environments we placed our users in, how we recruited our 124 participants, and the procedure for the experiment. 

\subsubsection{Setup}
We simulated the same three censorship environments from 
our qualitative user study. Details of the mild, intermediate, and advanced 
censorship environments and what configurations are necessary for a 
successful connection can be found in Section~\ref{sec:setup}. 

We instrumented Tor Launcher 5.0.3 and our redesigned Tor Launcher.
The instrumentation logs recorded every meaningful interaction with the interface---every
button press, every menu selection, every screen change---along with a timestamp.
The logs enabled us to calculate statistics such as the success rate
for each condition. We also recorded the particiapants' computer screens 
throughout the experiment to capture non-interface activity such as 
web searching and inspection of system networking settings.

We ran our experiment at Xlab, the Experimental Social Science Laboratory at University of 
California, Berkeley. The Xlab allows researchers to run experiments on 36 participants at a time.
There are 36 Windows machines in rows of 4, which are separated by cubicle walls. 
We provided the scripts to set up the simulated censorship environment, install necessary software, 
start the video recording, and save the logs and videos. Our experiment was not limited by Xlab
requirements. However, the configuration interface relies on OS-provided features but testing 
was only done on Windows machines, as a byproduct of using Xlab. 

\subsubsection{Recruitment}
We recruited about 20 users for each censorship environment
and interface combination, resulting in a total of 124 users. 
Data analysis considerations deem 20 participants in each 
6 interface version and simulated censorship combination a 
minimum to be able to assume normal distributions and 60 participants
testing each interface is large enough to have adequate effect size.

We recruited half of our users from Craigslist, and half of our participants from 
the Xlab participant pool. Although Xlab participants are not limited to UC Berkeley students and staff,
a majority of the participants are from campus. For this reason, we chose to recruit 
half of our participants from Craigslist to ensure a diverse set of participants. 
The recruitment text can be found in Appendix~\ref{quantitative-recruitment}. 

Out of our 124 participants, 59 were recruited from the Xlab pool and the other 65 were
recruited from Craiglist. Ages ranged from 18 to 68
(mu = 28.9, sigma = 12). 56.8\% were male and 
84.8\% of our participants had at least a college education.

\subsubsection{Procedure}
The one-hour, multi-participant procedure begins when all participants are sitting at their
respective computers in Xlab. Each computer is equipped with an old or modified version
of Tor Browser, Chrome, Firefox, Internet Explorer,  Chrome, and VLC (for screen recording).
Each computer was assigned one of the 6 conditions in the beginning of the study. Participants
were assigned to the seats at random, effectively being assigned a configuration launcher and
simulated censorship environment combination at random. 

Participants are firstly informed of the risks of the study and consenting to data collection.  At
this time, participants are given the chance to leave the laboratory if they do not consent to 
the experiment conditions. When every participant left in the room has turned in their consent
form, the experiment formally begins. A researcher informs the participants that they are in a
simulated censorship environment, where some websites and services are blocked. 
For the full script of what participants have heard, see Appendix~\ref{quantitative-script}. We
instruct them to visit a sample blocked website on a non-Tor browser of their choice to illustrate 
the situation.

After illustrating the censorship environment, participants are asked to 
complete a worksheet that asks to visit one blocked website. 
To mirror the qualitative study, we chose Wikipedia's featured article of the day 
as the blocked website. 

After instructions, researchers maintained minimal interactions with the participants, 
only answering logistical questions. Participants had
40 minutes to complete their worksheet. 
The participants do not know the details of their censorship environment,
only that they are actively being censored. Participants needed to configure Tor Browser to 
circumvent the simulated censorship. 

After users completed the browsing tasks, they took a short exit survey (Appendix~\ref{quantitative-exit-survey})
collecting their demographics. All users were instructed to sit until the end of the experiment,
regardless of when they had completed their task. After the 40 minutes were up, 
participants were officially informed that their time was up, and were given their payment of 
\$30 for their time. 

\subsubsection{Changes During the Experiment} 

% $ grep 'default_bridge\.' tor-browser_en-US/Browser/TorBrowser/Data/Browser/profile.default/preferences/extension-overrides.js | awk -F. '{print $4}' | sort | uniq -c
%       5 flashproxy
%       6 fte
%       2 fte-ipv6
%       1 meek-amazon
%       1 meek-azure
%       1 meek-google
%       5 obfs3
%       3 obfs4
%       2 scramblesuit
% The 5 flashproxy lines are effectively 1.
Tor Browser 5.0.3 came configured with default bridges:
1 flashproxy;
8 fte;
3 meek;
5 obfs3;
3 obfs4;
and 2 scramblesuit.
In the time since 5.0.3 was released and our later experiment,
two of the obfs4 bridges and one of the scramblesuit bridges stopped running.
% These are the bridges that stopped running. schanenlied is known and intentional.
% 188.226.213.208 in known and intentional: https://bugs.torproject.org/17318. Maybe
% it was not even running at the time of 5.0.3.
% mercurius4 is probably not running by accident.
% schanenlied:
% obfs4 178.209.52.110:443 67E72FF33D7D41BF11C569646A0A7B4B188340DF cert=Z+cv8z19Qb8RxWlkagp7SxiDQN++b7D2Tntowhf+j4D15/kLuj3EoSSGvuREGPc3h60Ofw iat-mode=0
% mercurius4:
% obfs4 104.131.108.182:56880 EF577C30B9F788B0E1801CF7E433B3B77792B77A cert=0SFhfDQrKjUJP8Qq6wrwSICEPf3Vl/nJRsYxWbg3QRoSqhl2EB78MPS2lQxbXY4EW1wwXA iat-mode=0
% (nickname unknown):
% scramblesuit 188.226.213.208:54278 AA5A86C1490296EF4FACA946CC5A182FCD1C5B1E password=MD2VRP7WXAMSG7MKIGMHI4CB4BMSNO7T
This left enough bridges that participants in the later experiment would be able to connect
under the same circumstances as those in the earlier experiment.
We did not modify the set of bridges between the experiments,
with one exception.
One of the meek bridges changed its bridge identity fingerprint;
Tor clients check the identity fingerprint and refuse to connect
if it does not match.
Therefore we patched the correct fingerprint in the later experiment.

\subsection{Results} 
Results from statistical analyses say: 
\begin{itemize} 
	\item{{\bfseries impact of censorship environment on success rate is not significant.} 
	Kruskal-Wallis, chi-squared = 4.7059, df = 2, p-value = 0.09509}
	\item{{\bfseries impact of censorship environment on time to completion is significant.} 
	Kruskal-Wallis, chi-squared = 64.565, df = 2, p-value = 9.547e-15}
	\item{{\bfseries impact of configuration version on success rate is not significant.}
	Mann-Whitney, W = 5.5, p-value = 0.8248}
	\item{{\bfseries impact of configuration interface on time to completion is not significant.} 
	Mann-Whitney, W = 991, p-value = 0.3197}
\end{itemize} 

A logistic regression model showing effects of censorship environment, interface version,
and recruitment pool on success finds that a difficult censorship environment is the most
significant factor in a participant failing. Xlab participants were more likely to succeed. 
Participants with the old version of the interface were more likely to fail. 

Coefficients:
             Estimate Std. Error z value Pr(>|z|)  
(Intercept)   19.4556  1684.3342   0.012   0.9908  
envE2        -17.3458  1684.3342  -0.010   0.9918  
envE3        -19.2652  1684.3341  -0.011   0.9909  
versionOLD    -0.7095     0.5793  -1.225   0.2206  
poolxlab       1.1218     0.5968   1.880   0.0601 .
---
Signif. codes:  0 ‘***’ 0.001 ‘**’ 0.01 ‘*’ 0.05 ‘.’ 0.1 ‘ ’ 1

A linear regression model showing effects of censorship environment, interface version, 
and recruitment pool on success finds that a difficult censorship environment is the most
significant factor in taking a long time to complete the task. Participants with the old version
of the interface and participants recruited from Xlab were likely to take longer. 

Coefficients:
            Estimate Std. Error t value Pr(>|t|)    
(Intercept)   -42.79      86.70  -0.494  0.62280    
envE2         300.65      90.13   3.336  0.00124 ** 
envE3         923.66     100.94   9.150 1.68e-14 ***
versionOLD    104.78      78.70   1.331  0.18642    
poolxlab       92.41      78.88   1.172  0.24445    
---
Signif. codes:  0 ‘***’ 0.001 ‘**’ 0.01 ‘*’ 0.05 ‘.’ 0.1 ‘ ’ 1

\begin{table}
\centering
	% Generated by info.R.
	\begin{tabular}{l r r r r}
	& \multicolumn{2}{c}{success rate} & \multicolumn{1}{c}{median time} \\
	& \multicolumn{2}{c}{after 40 minutes} & \multicolumn{1}{c}{to success} \\
	\noalign{\hrule}
	E1-NEW & 19/19 & 100\% & 0:20 \\
	E1-OLD & 19/19 & 100\% & 1:01 \\
	E2-NEW & 18/19 & 95\% & 3:22 \\
	E2-OLD & 16/19 & 84\% & 4:01 \\
	E3-NEW & 13/20 & 65\% & 12:38 \\
	E3-OLD & 10/20 & 50\% & 18:16 \\
	\end{tabular}
\caption{A summary of partipants' success in circumventing censorship
given their simulated censorship environment and version of Tor. Those who
failed to connect successfully were not used to calculate the median time
to success.}
\end{table}

\begin{figure}
\centering
\includegraphics{time_to_success}
\caption{
Time to first success, by censorship environment and interface.
The dots show the raw completion times;
while the boxplots show the medians and interquartile ranges.
The ``DNF'' figures at the right
show the number of participants who did not finish
in the time alloted.
}
\label{fig:time_to_success}
\end{figure}

%\begin{figure}
%\centering
%\includegraphics{time_to_success_clamped}
%\caption{
%Time to first success, by censorship environment and interface.
%Here, non-finishing participants are assigned a time of 40 minutes.
%}
%\label{fig:time_to_success_clamped}
%\end{figure}

\begin{figure}
\centering
\includegraphics{time_to_success_ecdf}
\caption{
Cumulative success rates over time, by censorship environment and interface.
We stopped participants after 40 minutes. Those who did not finish were assigned
an arbitrarily high number greater than 40 minutes. 
}
\label{fig:time_to_success_ecdf}
\end{figure}

\begin{table}
\centering 
	% Generated by info.R:
	\begin{tabular}{l r r r r}
	& \multicolumn{1}{c}{1.5~m} & \multicolumn{1}{c}{10~m} & \multicolumn{1}{c}{20~m} & \multicolumn{1}{c}{40~m} \\
	\noalign{\hrule}
	E1-NEW & 100\% & 100\% & 100\% & 100\% \\
	E1-OLD & 58\% & 100\% & 100\% & 100\% \\
	E2-NEW & 11\% & 79\% & 89\% & 95\% \\
	E2-OLD & 16\% & 68\% & 79\% & 84\% \\
	E3-NEW & 0\% & 25\% & 45\% & 65\% \\
	E3-OLD & 0\% & 20\% & 25\% & 50\% \\
	\end{tabular}
\caption{This Table shows what the success rate would have been
at different cutoff times.
For example, every E1-NEW participant finished within 90 seconds,
but only 58\% of E1-OLD had finished by that time.
After 10 minutes, 79\% of E2-NEW and 68\% of E2-OLD had finished.
After 20 minutes, 45\% of E3-NEW and only 25\% of E3-OLD had finished.}
\end{table}

\begin{figure*}
\centering
\includegraphics{all-participant-edges}
\caption{
Summary of participants' actions throughout the entire experiment.
Different colors indicate which screen was shown at each moment.
The overall length of the lines show the total time to completion,
except for those we cut off after approximately 40 minutes.
}
\label{fig:all-participant-edges}
\end{figure*}

\begin{figure} 
%\includegraphics{}
{\color {red} TODO} 
\caption{time spent on each screen}
\end{figure} 

\begin{figure} 
%\includegraphics{}
{\color {red} TODO} 
\caption{order pluggable transports were chosen}
\end{figure} 

\section{Discussion} 
{\color {red} TODO: improvements.} \\
{\color {red} TODO: subjective observations.} \\
{\color {red} TODO: remaining issues.} \\
{\color {red} recommendations are below.} \\

These non-conservative changes may be helpful for a user, and would benefit from future research: \\

\begin{itemize}
\item {\bfseries Tell people what to do.} Since a majority of the users will succeed if they click connect on the first screen, replacing the information about choices with explicit instructions to connect can lessen their cognitive load during the configuration process. 
\item{\bfseries Hide infrequently used options.} Hide the proxy screen and do not the option to configure a proxy unless it has been detected as necessary.
\item{\bfseries Tailor for at-risk users.} Currently, the interface is for the average user. We guide at-risk users to the correct configuration, but suggesting that they try the default first. However, this isn't ideal. 
\end{itemize} 

{\color {red} Speculation of why certain things happened--long progress bar (there was no feedback so people assumed that they had to wait), transport selections (people in the new interface didn't see the advice sometimes), etc.} 

\section{Limitations} 
The interface was only tested on Windows machines, which were the only types of machines in Xlab. The configuration interface uses the native operating system's elements and their respective styling, so an interface looks slightly different across different operating systems. Participants who are not accustomed to using Windows machines may have been slower than usual to complete the given task, but this effect should affect all of our conditions equally. 

Our study was conducted in an inorganic setting, which can cause our participants to be under or over-motivated. Our participants had a monetary incentive to connect to Tor, whereas a real user in a censored environment would want to reach a particular website or service. The nature of the experiment places social pressure to attempt connect for the whole duration of the session, whereas participants may have given up earlier if they were not in an organic setting.   

We did not test cases in which the user is required to configure a custom bridge or proxy. Our participants were from one geographic location in the United States and are able to speak English.

\section{Future Work} 
Throughout the qualitative user study, redesign of the interface, and the quantitative user study, we collaborated and communicated with Tor developers. Redesigning the configuration interface was a mutual interest. In fact, Tor version {\color{red} 5.?.?} incorporated textual and navigational changes based on our redesigned interface. We plan to continue our work with Tor developers to integrate changes we found to be helpful. This process will entail debugging our code, merging changes with the current version of Tor, making high-fidelity graphics that align with style guides, and finalizing the advice to give to users in the interface. 

We took an approach that optimized for the average case user yet was conservative enough in automation to allow users to connect without leaking that they use Tor. Additional user studies that focus specifically on users in heavily censored environments, or increasing the amount of automation in the configuration process can yield to higher rates of success and faster time to success. We encourage future work to explore some of the non-conservative changes and alternate approaches that we were not able to test in these series of experiments.\\

\begin{itemize}
\item{\bfseries Automate the configuration.} This sounds like a radical idea, but this wouldn't harm users today. Additionally, our study finds that most users would leak that they are using Tor when trying to circumvent an advanced censor. 
\item {\bfseries Automate after failure.} After an unsuccessful connection attempt, the user has already indicated that they are connecting to Tor on the network. Assuming that the significant difference is between being logged trying to connect to Tor or not at all, we can automate the entire configuration process thereafter without increasing risk. This may save our users a lot of headache. 
\item{\bfseries Ask about the risk.} Ask the users if they would be at risk if the process was automated, and automate the configuration process for all who are not at risk. The ones at risk will configure manually. The complication with this approach is that users may not be qualified to answer this question or may not trust Tor to answer honestly. 
\item{\bfseries Ask if users if they know what to do.} We ask users if they know how to configure a bridge and proxy, and automate the configuration process if they do not. However, there are ethical implications of making mistakes on behalf of the users if there is risk involved. 
\end{itemize}

Ideally, we want users in censorship environments across the world to succeed in connecting to the Tor network in a couple of minutes. We believe that this is possible with continued improvements to the interface. 

\section {Acknowledgments}
{\color {red} Rowilma del Castillo for setting up Xlab, Nima Fatemi, Isabela Bagueros, Georg Koppen, and the UX team for giving feedback, and Cecile Basnage for reviewing the UI of circumvention tools.} 
{\color {red} Tor Browser devs for taking our recommendations to heart and implementing changes.}

\section{Conclusion} 
{\color {red} I am a fan of not having long, rambling conclusions. I leave
that for the discussion section. This section is more of a courtesy to the 
people who read the abstract, look at the figures, and then read the 
conclusion. Think of something to tell those people here. Something along
the lines of a tl;dr summary of the paper.} 

\bibliographystyle{abbrv}
\nocite{*}
\bibliography{pets2017-paper}

\appendix
\section{Participant Worksheet Text} 
\label{participant-worksheet}
Imagine you live in an oppressive country that censors part of the Internet. We have simulated this in the laboratory by blocking certain websites and services. The purpose of this experiment is to evaluate the use of Tor browser, which is a browser that can circumvent censorship and let you visit blocked websites. For instance, www.torproject.org is blocked. Check this by going to the site on a standard browser, like Firefox, Chrome, or Internet Explorer. It will fail to load, when you can visit other sites.

To complete this worksheet, you will need to set up Tor browser (on your desktop) correctly and use it to get to blocked site. If you can visit wikipedia, then you know that you have successfully circumvented censorship.

\section{Qualitative User Study Recruitment Posting} 
\label{qualitative-recruitment}
We are recruiting participants for an in-person research study at the University of California, Berkeley. You will need to come in to our lab and perform tasks on a computer for an hour or less. You will be compensated \$30 for participating. 
No special knowledge and no technical experience is required. If you are interested, fill out the survey at \textit{<survey link>}. 

\section{Qualitative User Study Prescreening Survey} 
\label{qualitative-prescreening}
We are recruiting participants for an in-person research study at the University of California, Berkeley. You will need to come in to our lab and perform tasks on a computer for an hour or less. You will be compensated \$30 for participating. No special knowledge and no technical experience is required.\\

\begin{enumerate}
\item{Please select when you are available. We will assign you an hour experiment time slot during one of those times.}
\item{I am able to provide my own transportation to the University of California, Berkeley campus.}
\item{Thank you for your interest! Please provide an email address where we can contact you to share more logistical details.}
\item{we are looking for a very small number of participants, so unfortunately, we may not be able to accommodate everyone who applies. Would you like us to let you know about future opportunities?}
\item{What is your gender?}
\item{What is your age?}
\item{Please select your highest completed (or current) level of education.}
\item{What is your occupation?} 
\item{Do you speak any languages other than English fluently?}
\item{If you have a personal computer, what kind do you use?}
\item{Which of the following terms have you heard of? \textit{<answer choices: a checkboxlist of the the following terms: malware, proxy services, phishing, SSL, X.511 certificates, Tor>}}
\item{How often do you use the following software or features? \textit{<answer choices: a grid of radio buttons. Software/features (rows): HTTPS on web pages, proxies or other censorship circumvention tools, virtual private networks (VPN), file or whole-disk encryption, anonymity systems (e.g., Tor), email encryption (e.g., PGP), chat or instant messaging encryption, voice communication encryption. Frequency (columns): never, less than once a month, a few times a month, several times a week, daily.>}}
\end{enumerate}
Thank you for filling out this form. You are now done!

\section{Qualitative User Study Introduction Script} 
\label{qualitative-script} 
Imagine you live in an oppressive country that censors part of the Internet. We have simulated this in the laboratory by blocking certain websites and services.  The purpose of this experiment is to evaluate the use of Tor browser, which is a browser that can circumvent censorship and let you visit blocked websites. Currently, torproject is blocked (you can check this by going to torproject.org on a standard browser, like Firefox, Chrome, or Internet Explorer). 

To circumvent censorship successfully, you will need to set up Tor browser correctly and use it to get to Wikipedia. If you are able to reach the website, then you know that you have successfully circumvented censorship. Fill out the question on the worksheet. This isn't intended to be hard, just write what you see. We want to just check you saw the website. 

Before you start, do you have any questions about what you are asked to do? 

\section{Post-Experiment Standard Interview Questions}
We asked our participants these questions after they were given time to configure Tor Browser. \\

\begin{enumerate}
\item{Can you talk us through what you did along with what you were thinking at the time?}
\item{What was most challenging part of connecting?}
\item{Were there any unfamiliar terms?}
\item{How did you decide which options to choose?}
\item{What did you think about using Tor?}
\item{What is one change you would recommend?} 
\item{Did you need any additional information?} 
\end{enumerate}  

In addition to these questions, we asked our participants about specific questions based on their observation, usually regarding a specific choice in action, a particular screen they seemed stuck on, and any errors they encountered during the configuration process. 

\section{Quantitative User Recruitment Posting}
\label{quantitative-recruitment}
We are recruiting up to 40 participants for a user study at UC Berkeley. The experiment will involve basic Internet browsing tasks. You are not eligible if you have participated in our previous sessions.\\

\indent Payment: \$30 Amazon gift card\\
\indent Duration: 1 hour \\
\indent Where: Xlab at Hearst Memorial Gymnasium\\

\textit{<list of sessions>}\\

To be eligible, you must be an adult (18 or older). This is to comply with university policies on research. 

If you are interested: 1. Email lnl@berkeley.edu with the sessions you are able to attend. We will confirm your participation and assign you a session. 2. Come to Xlab at the appointed time for the experiment.

\section{Quantitative User Study Introduction Script} 
\label{quantitative-script} 
Imagine you live in an oppressive country that censors part of the Internet. We have simulated this in the laboratory by blocking certain websites and services.  The purpose of this experiment is to evaluate the use of Tor browser, which is a browser that can circumvent censorship and let you visit blocked websites. Currently, torproject is blocked (you can check this by going to torproject.org on a standard browser, like Firefox, Chrome, or Internet Explorer). 

To circumvent censorship successfully, you will need to set up Tor browser correctly and use it to get to Wikipedia. If you are able to reach the website, then you know that you have successfully circumvented censorship. Fill out the question on the worksheet. This isn't intended to be hard, just write what you see. We want to just check you saw the website. 

Afterward, we ask you to take a short survey to collect some information about you. The link is also on your worksheet.
We will give you time to complete this task. If you finish early, we ask that you sit at your desk until the remainder of the hour. Since we are recording your screen, we ask that you don't do anything personal afterward, like checking your email.

Before you start, do you have any questions about what you are asked to do? 

\section{Quantitative User Study Exit Survey} 
\label{quantitative-exit-survey}
We'd like to know more about you.  All of your answers will be stored separately from any identifying information in order to protect your confidentiality.

This survey is part of a research project being conducted by the University of California, Berkeley. If you have any questions about your rights or treatment as a research participant in this study, please contact the University of California at Berkeley's Committee for Protection of Human Subjects at 510-642-7461, or email subjects@berkeley.edu. If you agree to participate, please click Next below.\\

\begin{enumerate}
\item{What is your participant ID? (This can be found on the sticker ont he left hand corner of the desk you are currently sitting at.)}
\item{What is your gender?}
\item{What is your age?}
\item{Please select your highest completed (or current) level of education}.
\item{What is your current occupation?}  
\end{enumerate}

Thank you for participating in our experiment. You are now done! Please sit at your desk for the remainder of the experiment. Our researchers will formally announce the end of the experiment. 
\end{document}
